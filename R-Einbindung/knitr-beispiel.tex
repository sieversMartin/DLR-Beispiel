%%% Aufruf mit: "PATH/TO/RScript.exe" -e "library(knitr); knit2pdf('%.Rnw')"

\documentclass{book}\usepackage[]{graphicx}\usepackage[]{color}
%% maxwidth is the original width if it is less than linewidth
%% otherwise use linewidth (to make sure the graphics do not exceed the margin)
\makeatletter
\def\maxwidth{ %
  \ifdim\Gin@nat@width>\linewidth
    \linewidth
  \else
    \Gin@nat@width
  \fi
}
\makeatother

\definecolor{fgcolor}{rgb}{0.345, 0.345, 0.345}
\newcommand{\hlnum}[1]{\textcolor[rgb]{0.686,0.059,0.569}{#1}}%
\newcommand{\hlstr}[1]{\textcolor[rgb]{0.192,0.494,0.8}{#1}}%
\newcommand{\hlcom}[1]{\textcolor[rgb]{0.678,0.584,0.686}{\textit{#1}}}%
\newcommand{\hlopt}[1]{\textcolor[rgb]{0,0,0}{#1}}%
\newcommand{\hlstd}[1]{\textcolor[rgb]{0.345,0.345,0.345}{#1}}%
\newcommand{\hlkwa}[1]{\textcolor[rgb]{0.161,0.373,0.58}{\textbf{#1}}}%
\newcommand{\hlkwb}[1]{\textcolor[rgb]{0.69,0.353,0.396}{#1}}%
\newcommand{\hlkwc}[1]{\textcolor[rgb]{0.333,0.667,0.333}{#1}}%
\newcommand{\hlkwd}[1]{\textcolor[rgb]{0.737,0.353,0.396}{\textbf{#1}}}%

\usepackage{framed}
\makeatletter
\newenvironment{kframe}{%
 \def\at@end@of@kframe{}%
 \ifinner\ifhmode%
  \def\at@end@of@kframe{\end{minipage}}%
  \begin{minipage}{\columnwidth}%
 \fi\fi%
 \def\FrameCommand##1{\hskip\@totalleftmargin \hskip-\fboxsep
 \colorbox{shadecolor}{##1}\hskip-\fboxsep
     % There is no \\@totalrightmargin, so:
     \hskip-\linewidth \hskip-\@totalleftmargin \hskip\columnwidth}%
 \MakeFramed {\advance\hsize-\width
   \@totalleftmargin\z@ \linewidth\hsize
   \@setminipage}}%
 {\par\unskip\endMakeFramed%
 \at@end@of@kframe}
\makeatother

\definecolor{shadecolor}{rgb}{.97, .97, .97}
\definecolor{messagecolor}{rgb}{0, 0, 0}
\definecolor{warningcolor}{rgb}{1, 0, 1}
\definecolor{errorcolor}{rgb}{1, 0, 0}
\newenvironment{knitrout}{}{} % an empty environment to be redefined in TeX

\usepackage{alltt}
\usepackage[utf8]{inputenc}
\usepackage[T1]{fontenc}
\usepackage[ngerman]{babel}
\usepackage{libertine}
\usepackage[varqu]{inconsolata}
\IfFileExists{upquote.sty}{\usepackage{upquote}}{}
\begin{document}

    
\begin{knitrout}
\definecolor{shadecolor}{rgb}{0.969, 0.969, 0.969}\color{fgcolor}\begin{kframe}
\begin{alltt}
\hlstd{X} \hlkwb{<-} \hlkwd{read.csv}\hlstd{(}\hlstr{"mydata.csv"}\hlstd{,}\hlkwc{header}\hlstd{=T)}
\hlkwd{summary}\hlstd{(X)}
\end{alltt}
\begin{verbatim}
##   Schüler.Nr.    Note.Mädchen  Note.Jungen 
##  Min.   : 1.0   Min.   :1.2   Min.   :1.0  
##  1st Qu.: 3.2   1st Qu.:2.6   1st Qu.:2.0  
##  Median : 5.5   Median :2.8   Median :2.6  
##  Mean   : 5.5   Mean   :2.8   Mean   :3.0  
##  3rd Qu.: 7.8   3rd Qu.:3.2   3rd Qu.:3.4  
##  Max.   :10.0   Max.   :3.5   Max.   :6.0
\end{verbatim}
\begin{alltt}
\hlstd{meanM} \hlkwb{<-} \hlkwd{mean}\hlstd{(X}\hlopt{$}\hlstd{Note.Jungen)}
\hlstd{meanF} \hlkwb{<-} \hlkwd{mean}\hlstd{(X}\hlopt{$}\hlstd{Note.Mädchen)}
\hlstd{stdabw} \hlkwb{<-} \hlkwa{function}\hlstd{(}\hlkwc{x}\hlstd{) \{n}\hlkwb{=}\hlkwd{length}\hlstd{(x) ;} \hlkwd{sqrt}\hlstd{(}\hlkwd{var}\hlstd{(x)} \hlopt{*} \hlstd{(n}\hlopt{-}\hlnum{1}\hlstd{)} \hlopt{/} \hlstd{n)\}}
\hlstd{stdabwM} \hlkwb{<-} \hlkwd{stdabw}\hlstd{(X}\hlopt{$}\hlstd{Note.Jungen)}
\hlstd{stdabwF} \hlkwb{<-} \hlkwd{stdabw}\hlstd{(X}\hlopt{$}\hlstd{Note.Mädchen)}
\end{alltt}
\end{kframe}
\end{knitrout}
    
Der Mittelwert der Noten der Jungen beträgt 3, der der Mädchen 2.8. Die 
Standardabweichung beträgt dabei 1.69 bzw. 0.62	

\begin{knitrout}
\definecolor{shadecolor}{rgb}{0.969, 0.969, 0.969}\color{fgcolor}\begin{kframe}
\begin{alltt}
\hlkwd{plot}\hlstd{(X}\hlopt{$}\hlstd{Note.Jungen)}
\end{alltt}
\end{kframe}

{\centering \includegraphics[width=\maxwidth]{figure/minimal-myplot-1} 

}



\end{knitrout}
\end{document}
